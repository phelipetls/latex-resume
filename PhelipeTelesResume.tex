\documentclass[11pt,letterpaper]{article}

\usepackage[left=0.4 in,top=0.4in,right=0.4 in,bottom=0.4in]{geometry}
\usepackage[utf8]{inputenc}
\usepackage[brazil]{babel}
\usepackage{hyperref}
\usepackage[parfill]{parskip}
\usepackage{array}

\newenvironment{tightlist}
  {\begin{list}
    {$\cdot$}
    {
      \setlength{\leftmargin}{0em}
      \setlength{\itemsep}{-0.5em}
    }
  }
{\end{list}}

\begin{document}

\centerline{\huge\bf PHELIPE TELES DA SILVA}
\medskip

\centerline{30 de jan. de 1996 $\diamond$ Rio de Janeiro}
\smallskip

\centerline{
  (21) 9 8047-3484 $\diamond$
  \href{mailto:phelipe_teles@hotmail.com}{phelipe\_teles@hotmail.com} $\diamond$ \href{https://linkedin.com/in/phelipeteles}{LinkedIn}
}
\smallskip

\medskip \textbf{EXPERIÊNCIA} \medskip
\hrule

\textbf{Mutual} \hfill Dez. 2020 -- Atualmente \\
\emph{Desenvolvedor front-end} \hfill \emph{Remoto} \\

\vspace*{-\baselineskip}

\begin{tightlist}
  \item Desenvolvimento de aplicações mobile e web com React, React Native,
    JavaScript e TypeScript.
  \item Implementação de testes de integração com Jest, react-testing-library,
    Cypress e Mock Service Worker.
  \item Melhoria do deploy do app mobile com CodePush, App Center e Gitlab CI/CD.
  \item Monitoramento de erros com Sentry.
  \item Uso de bibliotecas como Redux, Material UI, react-router, react-query,
    react-navigation etc.
  \item Component-driven development com Storybook.
\end{tightlist}

\medskip \textbf{PROJETOS} \medskip
\hrule

\begin{tightlist}
 \item \href{https://phelipetls.github.io}{\textbf{Blog}}: Blog desenvolvido com
   Hugo, HTML, JavaScript, TailwindCSS e GitHub Actions.
 \item \href{http://ipeadata-explorer.surge.sh}{\textbf{Ipeadata Explorer}}:
   Frontend para o Ipeadata. Uso de React, Material-UI, Cypress, Chart.js,
   react-simple-maps.
 \item \href{https://github.com/phelipetls/minesweeper.js}{\textbf{Campo minado}}:
   Express e PostgreSQL no backend. JavaScript e CSS puro no frontend.
 \item \href{https://github.com/phelipetls/seriesbr}{\textbf{seriesbr}}:
   Biblioteca em Python para consumir APIs do BCB, IPEA e IBGE em um DataFrame
   do pandas.
\end{tightlist}

\medskip \textbf{FORMAÇÃO} \medskip
\hrule

\textbf{Universidade Federal Rural do Rio de Janeiro} \hfill {Mar. 2015 -- Dez. 2020} \\
\emph{Ciências Econômicas} \hfill \\

\medskip \textbf{IDIOMAS} \medskip
\hrule

\begin{tightlist}
  \item Inglês avançado.
\end{tightlist}

\end{document}
