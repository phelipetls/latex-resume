\documentclass[11pt,letterpaper]{article}

\usepackage[left=0.4 in,top=0.4in,right=0.4 in]{geometry}
\usepackage[utf8]{inputenc}
\usepackage[brazil]{babel}
\usepackage{hyperref}
\usepackage[parfill]{parskip}
\usepackage{array}

\newenvironment{tightlist}
  {\begin{list}
    {$\cdot$}
    {
      \setlength{\leftmargin}{0em}
      \setlength{\itemsep}{-0.5em}
    }
  }
{\end{list}}

\begin{document}

\pagestyle{empty}

\centerline{\huge\bf PHELIPE TELES DA SILVA}
\medskip

\centerline{30 de jan. de 1996 $\diamond$ Rio de Janeiro}
\smallskip

\centerline{
  (21) 9 8047-3484 $\diamond$
  \href{mailto:phelipe_teles@hotmail.com}{phelipe\_teles@hotmail.com} $\diamond$ \href{https://linkedin.com/in/phelipeteles}{LinkedIn}
}
\smallskip

\medskip \textbf{EXPERIÊNCIA} \medskip
\hrule

\textbf{Mutual} \hfill Dez. 2020 -- Atualmente \\
\emph{Desenvolvedor front-end} \hfill \emph{Remoto} {\parfillskip=0pt\par}

\begin{tightlist}
  \item Ajudei a desenvolver e manter aplicações web e mobile com React, React
    Native e TypeScript.
  \item Construção de Single Page Applications com create-react-app e
    react-app-rewired e de sites estáticos com Next.js.
  \item Contribuí para maior implementação de testes unitários e de integração
    com react-testing-library, Mock Service Worker e Cypress.
  \item Melhorei a observabilidade de erros dos aplicativos mobile no Sentry, ao
    automatizar a geração e envio de source maps após um deploy de CodePush.
  \item Desenvolvimento orientado a componentes com Storybook.
  \item Experiência colaborando usando git, monorepos administrado com Lerna e
    GitLab.
\end{tightlist}

\medskip \textbf{PROJETOS} \medskip
\hrule

\textbf{\href{https://phelipetls.github.io}{Blog}} -- Hugo, JavaScript, Tailwind CSS, HTML, GitHub Actions, Cypress \hfill 2019
{\parfillskip=0pt\par}
Blog onde compartilho o que aprendo e meus interesses.

\medskip

\textbf{\href{http://ipeadata-explorer.surge.sh}{Ipeadata Explorer}} -- React, create-react-app, Material UI, Chart.js, Cypress \hfill 2020
{\parfillskip=0pt\par}
Aplicação para visualizar e explorar a base de dados de séries temporais
  do Ipea, o Ipeadata.

\medskip

\textbf{\href{https://github.com/phelipetls/seriesbr}{seriesbr}} -- Python, pandas, requests, pytest, poetry, GitHub Actions \hfill 2020
{\parfillskip=0pt\par}
Biblioteca em Python para consumir APIs do BCB, IPEA e IBGE em um DataFrame do pandas.

\medskip \textbf{FORMAÇÃO} \medskip
\hrule

\textbf{Universidade Federal Rural do Rio de Janeiro} \hfill {Mar. 2015 -- Dez. 2020} \\
\emph{Ciências Econômicas}

\medskip \textbf{IDIOMAS} \medskip
\hrule

\begin{tightlist}
  \item Inglês avançado.
\end{tightlist}

\end{document}
