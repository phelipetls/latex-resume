\documentclass[12pt]{article}

\usepackage[left=1cm,top=1cm,right=1cm]{geometry}
\usepackage[utf8]{inputenc}
\usepackage[brazil]{babel}
\usepackage{hyperref}
\usepackage[parfill]{parskip}
\usepackage{array}
\usepackage{palatino}
\usepackage[T1]{fontenc}
\usepackage{titlesec}
\usepackage{fontawesome5}

\titleformat
{\section} % command
[display] % shape
{\bfseries\large} % format
{} % label
{\medskip} % sep
{} % before-code
[\hrule] % after-code

\newenvironment{tightlist}
  {\begin{list}
    {$\cdot$}
    {
      \setlength{\leftmargin}{0em}
      \setlength{\itemsep}{\smallskipamount}
    }
  }
{\end{list}}

\begin{document}

\pagestyle{empty}

\centerline{\huge\bf PHELIPE TELES DA SILVA}
\medskip

\centerline{30 de jan. de 1996 $\cdot$ Rio de Janeiro}
\smallskip

\centerline{
  \href{mailto:telesphelipe@gmail.com}{\faIcon[solid]{envelope} E-mail}
  $\cdot$
  \href{https://phelipetls.github.io}{\faIcon[solid]{globe} Site pessoal}
  $\cdot$
  \href{https://linkedin.com/in/phelipeteles}{\faIcon[solid]{linkedin} LinkedIn}
}
\smallskip

\section*{EXPERIÊNCIA}

\textbf{\href{https://mutual.club}{Mutual}} \hfill Dez. 2020 -- Presente \\
\textit{Desenvolvedor front-end} \hfill \textit{Remoto} {\parfillskip=0pt\par}

Mutual é uma fintech no ramo de empréstimo peer-to-peer, em que trabalhei no
\href{https://mutual.club/en/invest.html}{aplicativo mobile para invesidores} e
no \href{https://scora.com.br/}{Scora}, um aplicativo web para refinancimaneto
de dívidas.

\begin{tightlist}
  \item Desenvolvi aplicativos para web com React e TypeScript, em sua maioria
    Single Page Apps com create-react-app e sites estáticos com Next.js.
  \item Desenvolvi aplicativos mobile com React Native, distribuídos com App
    Center e CodePush, e integrados com serviços para analytics (Facebook,
    AppsFlyer, Firebase), monitoramento (Sentry, App Center), push notifications
    (Firebase Cloud Messaging) e outros.
  \item Contribuí para uma cultura mais orientada a testes no front-end, ao
    escrever testes de integração com react-testing-library, Mock Service Worker
    e Cypress, e fomentando discussões a respeito em nossos chapters.
  \item Desenvolvi pipelines de CI/CD com GitLab CI/CD e shell scripts.
  \item Aumentei a velocidade na resolução de bugs ao automatizar o upload de
    source maps para o Sentry nas pipelines de CI/CD dos aplicativos mobile e
    web.
\end{tightlist}

\section*{TECNOLOGIAS}

\begin{description}
  \item[Linguagens de programação] JavaScript, TypeScript, Python, Bash, Lua.
  \item[Front-end] React, React Native, Next.js, create-react-app,
    Storybook, Jest, react-testing-library, webpack, Vite,
    MUI, Redux, react-query, Tailwind CSS.
  \item[Back-end] Express, Koa, Flask, MongoDB, PostgreSQL.
  \item[DevOps] Docker, GitLab CI/CD, GitHub Actions, Ansible.
\end{description}

\section*{FORMAÇÃO}

\textbf{Universidade Federal Rural do Rio de Janeiro} \hfill {Mar. 2015 -- Dez. 2020} \\
\textit{Ciências Econômicas} \hfill \textit{Seropédica, RJ} {\parfillskip=0pt\par}

\section*{IDIOMAS}

\begin{tightlist}
  \item Inglês avançado.
\end{tightlist}

\end{document}
