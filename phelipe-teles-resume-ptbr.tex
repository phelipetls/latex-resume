\documentclass[12pt]{article}

\usepackage[left=1cm,top=1cm,right=1cm]{geometry}
\usepackage[utf8]{inputenc}
\usepackage[brazil]{babel}
\usepackage{hyperref}
\usepackage[parfill]{parskip}
\usepackage{array}
\usepackage{palatino}
\usepackage[T1]{fontenc}

\newenvironment{tightlist}
  {\begin{list}
    {$\cdot$}
    {
      \setlength{\leftmargin}{0em}
      \setlength{\itemsep}{-\smallskipamount}
    }
  }
{\end{list}}

\begin{document}

\pagestyle{empty}

\centerline{\huge\bf PHELIPE TELES DA SILVA}
\medskip

\centerline{30 de jan. de 1996 $\cdot$ Rio de Janeiro}
\smallskip

\centerline{
  \href{mailto:phelipe_teles@hotmail.com}{phelipe\_teles@hotmail.com}
  $\cdot$
  \href{https://linkedin.com/in/phelipeteles}{LinkedIn}
}
\smallskip

\medskip \textbf{EXPERIÊNCIA} \medskip
\hrule

\textbf{Mutual} \hfill Dez. 2020 -- Presente \\
\emph{Desenvolvedor front-end} \hfill \emph{Remoto} {\parfillskip=0pt\par}

Mutual é uma fintech no ramo de empréstimo peer-to-peer, em que trabalhei
principalmente mantendo o \href{https://mutual.club/en/invest.html}{app para
invesidores}, escrito com React Native e TypeScript, e no
\href{https://scora.com.br/}{Scora}, um aplicativo para web para refinancimaneto
de dívidas, escrito com React.

\medskip

\begin{tightlist}
  \item Desenvolvi aplicativos para web com React e TypeScript, em sua maioria
    Single Page Apps com create-react-app e sites estáticos com Next.js.
  \item Me tornei proficiente em bibliotecas e ferramentas comuns no ecossistema
    React, como Storybook, Redux, react-router, @mui/material, react-hook-form,
    react-query etc..
  \item Mantive aplicativos móveis escritos com React Native, distribuídos com
    App Center e CodePush, e integrados com serviços para analytics (Facebook,
    AppsFlyer, Firebase), monitoramento (Sentry, App Center), e cloud messaging
    (Firebase Cloud Messaging).
  \item Contribuí para uma cultura mais orientada a testes no front-end,
    ao escrever testes de integração com react-testing-library, Mock Service
    Worker e Cypress, e falando sobre em nossos chapters.
  \item Ganhei experiência contribuindo e revisando código no GitLab.
  \item Escrevi pipelines de CI/CD com GitLab CI/CD e scripts de shell.
  \item Aumentei a velocidade na resolução de bugs ao automatizar o upload de
    source maps para o Sentry nas pipelines de CI/CD dos aplicativos mobile e
    web.
\end{tightlist}

\medskip \textbf{FORMAÇÃO} \medskip
\hrule

\textbf{Universidade Federal Rural do Rio de Janeiro} \hfill {Mar. 2015 -- Dez. 2020} \\
\emph{Ciências Econômicas} \hfill \emph{Seropédica, RJ} {\parfillskip=0pt\par}

\medskip \textbf{IDIOMAS} \medskip
\hrule

\begin{tightlist}
  \item Inglês avançado.
\end{tightlist}

\end{document}
