\documentclass[12pt,letterpaper]{article}

\usepackage[left=0.4 in,top=0.4in,right=0.4 in]{geometry}
\usepackage[utf8]{inputenc}
\usepackage[brazil]{babel}
\usepackage{hyperref}
\usepackage[parfill]{parskip}
\usepackage{array}

\newenvironment{tightlist}
  {\begin{list}
    {$\cdot$}
    {
      \setlength{\leftmargin}{0em}
      \setlength{\itemsep}{-\smallskipamount}
    }
  }
{\end{list}}

\begin{document}

\pagestyle{empty}

\centerline{\huge\bf PHELIPE TELES DA SILVA}
\medskip

\centerline{30 de jan. de 1996 $\cdot$ Rio de Janeiro}
\smallskip

\centerline{
  \href{mailto:phelipe_teles@hotmail.com}{phelipe\_teles@hotmail.com}
  $\cdot$
  \href{https://linkedin.com/in/phelipeteles}{LinkedIn}
}
\smallskip

\medskip \textbf{EXPERIÊNCIA} \medskip
\hrule

\textbf{Mutual} \hfill Dez. 2020 -- Atualmente \\
\emph{Desenvolvedor front-end} \hfill \emph{Remoto} {\parfillskip=0pt\par}

\begin{tightlist}
  \item Desenvolvimento de aplicações web com React, create-react-app e Next.js.
  \item Desenvolvimento de aplicações mobile com react-native-cli e workflow bare do
    Expo, usando App Center e CodePush para distribuição.
  \item Familiaridade com ferramentas e bibliotecas de front-end como
    Stroybook, react-router, @mui/material, react-hook-form, yup,
    react-query, Redux e outras.
  \item Contribuí para uma cultura mais orientada a testes, ao escrever e
    ensinar sobre testes unitários e de integração com
    react-testing-library, Mock Service Worker e Cypress.
  \item Melhorei o monitoramento de erros das aplicações mobile ao aumentar a
    legibilidade do código de produção no Sentry com a ajuda de source
    maps, o que aumentou nossa velocidade na resolução de bugs.
  \item Experiência colaborando e revisando código usando GitLab.
\end{tightlist}

\medskip \textbf{FORMAÇÃO} \medskip
\hrule

\textbf{Universidade Federal Rural do Rio de Janeiro} \hfill {Mar. 2015 -- Dez. 2020} \\
\emph{Ciências Econômicas} \hfill \emph{Seropédica, RJ} {\parfillskip=0pt\par}

\medskip \textbf{IDIOMAS} \medskip
\hrule

\begin{tightlist}
  \item Inglês avançado.
\end{tightlist}

\end{document}
