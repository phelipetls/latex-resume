\documentclass{resume}

\usepackage[left=0.4 in,top=0.4in,right=0.4 in,bottom=0.4in]{geometry}
\usepackage[utf8]{inputenc}
\usepackage[brazil]{babel}
\usepackage{hyperref}
\usepackage{calcage}

\newcommand{\tab}[1]{\hspace{.2667\textwidth}\rlap{#1}}
\newcommand{\itab}[1]{\hspace{0em}\rlap{#1}}
\name{PHELIPE TELES DA SILVA}
\address{30 de jan. de 1996 (\calcage{1996}{01}{30}anos) \\ Rio de Janeiro}
\address{
  (21) 9 6678-8995 \\ phelipe\_teles@hotmail.com \\
  \href{https://linkedin.com/in/phelipeteles}{linkedin.com/in/phelipeteles} \\
  \href{https://phelipetls.github.io}{phelipetls.github.io}
}

\begin{document}

%----------------------------------------------------------------------------------------
%   EDUCATION SECTION
%----------------------------------------------------------------------------------------

\begin{rSection}{FORMAÇÃO}
  {\bf Ciências Econômicas} \hfill {Mar 2015 -- Dez  2020}
  \\
  Universidade Federal Rural do Rio de Janeiro.
  \\
\end{rSection}

%----------------------------------------------------------------------------------------
%   WORK EXPERIENCE SECTION
%----------------------------------------------------------------------------------------

\begin{rSection}{EXPERIÊNCIA}
  \begin{rSubsection}{Desenvolvedor front-end}{Dez 2020 -- atualmente}{}
  \item
  \item Desenvolvimento de aplicações mobile e web com React, React Native e
    TypeScript.
  \item Implementação de testes de integração com react-testing-library e
    Cypress.
  \item Melhoria do processo de release com CodePush, App Center e monitoramento
    de erros com Sentry, usando Gitlab CI/CD.
  \item Experiência com Redux, Material-UI, react-hook-form, etc.
  \vspace{5mm}
  \end{rSubsection}
\end{rSection}

%----------------------------------------------------------------------------------------
%   TECHNICAL STRENGTHS SECTION
%----------------------------------------------------------------------------------------

\begin{rSection}{SOFTWARES \& LINGUAGENS DE PROGRAMAÇÃO}
  \begin{tabular}{ @{} >{\bfseries}l @{\hspace{6ex}} l }
    Básico & SQL.\\
    Intermediário & TypeScript, CSS, Python, Bash, YAML.\\
    Avançado & JavaScript, Git, HTML.\\
  \end{tabular}
  \vspace{5mm}
\end{rSection}

%----------------------------------------------------------------------------------------
%   PROJECTS SECTION
%----------------------------------------------------------------------------------------

\begin{rSection}{PROJETOS}
  \begin{rSubsection}{Desenvolvimento web}{}{}
  \item
  \item \href{https://github.com/phelipetls/minesweeper.js}{\textbf{Campo
    minado}}: Express.js e PostgreSQL no backend, JavaScript e CSS no
    frontend.
  \item \href{https://phelipetls.github.io}{\textbf{Site pessoal}}: Site pessoal
    desenvolvido com Hugo.
  \item \href{http://ipeadata-explorer.surge.sh}{\textbf{Ipeadata Explorer}}:
    Frontend escrito em React para o Ipeadata, usando Material-UI e testado com
    Cypress.
    \vspace{5mm}
  \end{rSubsection}

  \begin{rSubsection}{Pacotes no Python}{}{}
  \item
  \item \href{https://github.com/phelipetls/seriesbr}{\textbf{seriesbr}}:
    Biblioteca em Python para consumir séries de bases de dados do Banco
    Central, IPEA e IBGE em dataframes do pandas.
  \item \href{https://github.com/phelipetls/reportforce}{\textbf{reportforce}}:
    Biblioteca em Python para consumir relatórios do Salesforce usando a
    Analytics REST API.
    \vspace{5mm}
  \end{rSubsection}
\end{rSection}

%----------------------------------------------------------------------------------------
%   IDIOMS SECTION
%----------------------------------------------------------------------------------------

\begin{rSection}{IDIOMAS} \itemsep -3pt
    {Inglês avançado.}
\end{rSection}

\end{document}

% vi: nowrap
