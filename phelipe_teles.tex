\documentclass{resume}

\usepackage[left=0.4 in,top=0.4in,right=0.4 in,bottom=0.4in]{geometry}
\usepackage[utf8]{inputenc}
\usepackage[brazil]{babel}
\usepackage{hyperref}
\usepackage{calcage}

\name{PHELIPE TELES DA SILVA}
\address{30 de jan. de 1996 (\calcage{1996}{01}{30}anos) \\ Rio de Janeiro}
\address{
  (21) 9 8047-3484 \\ phelipe\_teles@hotmail.com \\
  \href{https://linkedin.com/in/phelipeteles}{linkedin.com/in/phelipeteles} \\
  \href{https://phelipetls.github.io}{phelipetls.github.io}
}

\begin{document}

%----------------------------------------------------------------------------------------
%   EDUCATION SECTION
%----------------------------------------------------------------------------------------

\begin{rSection}{FORMAÇÃO}
  \begin{rSubsection}{Universidade Federal Rural do Rio de Janeiro}{Mar. 2015 --
    Dez. 2020}{Ciências Econômicas}{}
  \end{rSubsection}
\end{rSection}

%----------------------------------------------------------------------------------------
%   WORK EXPERIENCE SECTION
%----------------------------------------------------------------------------------------

\begin{rSection}{EXPERIÊNCIA}
  \begin{rSubsection}{Mutual}{Dez. 2020 -- atualmente}{Desenvolvedor
    front-end}{Remoto}
  \item Desenvolvimento de aplicações mobile e web com React, React Native,
    JavaScript e TypeScript.
  \item Implementação de testes de integração com Jest, react-testing-library,
    Cypress e Mock Service Worker.
  \item Melhoria do deploy do app mobile com CodePush, App Center e Gitlab CI/CD.
  \item Monitoramento de erros com Sentry.
  \item Uso de bibliotecas como Redux, Material UI, react-router, react-query,
    react-navigation etc.
  \item Component-driven development com Storybook.
  \end{rSubsection}
\end{rSection}

%----------------------------------------------------------------------------------------
%   PROJECTS SECTION
%----------------------------------------------------------------------------------------

\begin{rSection}{PROJETOS}
  \begin{rSubsection}{}{}{}{}
  \item \href{https://phelipetls.github.io}{\textbf{Blog}}: Blog
    desenvolvido com Hugo, HTML, JavaScript, TailwindCSS e GitHub Actions.
  \item \href{http://ipeadata-explorer.surge.sh}{\textbf{Ipeadata Explorer}}:
    Frontend para a API do Ipeadata. Uso de React, Material-UI, Cypress,
    Chart.js, react-simple-maps.
  \item \href{https://github.com/phelipetls/minesweeper.js}{\textbf{Campo
    minado}}: Express e PostgreSQL no backend. JavaScript e CSS puro no
    frontend.
  \item \href{https://github.com/phelipetls/seriesbr}{\textbf{seriesbr}}:
    Biblioteca em Python para consumir APIs do BCB, IPEA e IBGE em um DataFrame
    do pandas.
  \end{rSubsection}
\end{rSection}

%----------------------------------------------------------------------------------------
%   IDIOMS SECTION
%----------------------------------------------------------------------------------------

\begin{rSection}{IDIOMAS}
  Inglês avançado.
\end{rSection}

\end{document}
